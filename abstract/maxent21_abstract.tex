%% LyX 2.3.4.2 created this file.  For more info, see http://www.lyx.org/.
%% Do not edit unless you really know what you are doing.
\documentclass[12pt]{article}
\usepackage[utf8]{inputenc}
\pagestyle{empty}
\usepackage{amsmath}

\makeatletter
\@ifundefined{date}{}{\date{}}
%%%%%%%%%%%%%%%%%%%%%%%%%%%%%% User specified LaTeX commands.
%%%%%%%%%%%%%%%%%%%%%%%%%%%%%%%%%%%%%%%%%%%%%%%%%%%%%%%%%%%%%%%%%%%%
%%%%%                                                          %%%%%
%%%%%  LaTeX macro for subscription of abstracts to            %%%%%
%%%%%                                                          %%%%%
%%%%%               MAXENT 2019                                %%%%%
%%%%%                                                          %%%%%
%%%%%                                                          %%%%%
%%%%% The total length of the contribution (including figures, %%%%%
%%%%% references, etc.) must fit in a 16.0 cm x 23.0 cm frame  %%%%%
%%%%% at 12pt type.                                            %%%%%
%%%%% If figures are included they must be supplied as         %%%%%
%%%%% Postscript files attached at the end of this document    %%%%%
%%%%% stating clearly the file name.                           %%%%%
%%%%%                                                          %%%%%
%%%%% Please respect the blank lines.                          %%%%%
%%%%%                                                          %%%%%
%%%%% Your name and affiliation (including e-mail and fax):    %%%%%
%%%%%                                                          %%%%%
%%%%%                                                          %%%%%
%%%%%                                                          %%%%%
%%%%% You are requesting a (choose one):                       %%%%%
%%%%%                                                          %%%%%
%%%%%                                                          %%%%%
%%%%% ( )    Oral presentation                                 %%%%%
%%%%% ( )    Poster                                            %%%%%
%%%%%                                                          %%%%%
%%%%%                                                          %%%%%
%%%%%                                                          %%%%%
%%%%%%%%%%%%%%%%%%%%%%%%%%%%%%%%%%%%%%%%%%%%%%%%%%%%%%%%%%%%%%%%%%%%



%%%%%%%%%%%%%%%%%%%%%%%%%%%%%%%%%%%%%%%%%%%%%%%%%%%%%%%%%%%%%%%%%%%%
%%%%%                                                          %%%%%
%%%%%  If you DO NOT need to include graphics you might        %%%%%
%%%%%  delete the above line                                   %%%%%
%%%%%                                                          %%%%%
%%%%%%%%%%%%%%%%%%%%%%%%%%%%%%%%%%%%%%%%%%%%%%%%%%%%%%%%%%%%%%%%%%%%



\textwidth16cm
\textheight23cm
\topmargin-2.0cm
\oddsidemargin0cm
\parindent0pt

\makeatother

\begin{document}
\title{SURROGATE-ENHANCED PARAMETER INFERENCE FOR FUNCTION-VALUED MODELS}
\author{\underline{C.G. Albert}$^{1}$, U. Callies$^{2}$, U. v. Toussaint$^{1}$\\
 (1) Max-Planck-Institut für Plasmaphysik, 85748 Garching, Germany\\
 (2) Helmholtz-Zentrum Hereon, 21502 Geesthacht, Germany\\
 albert@alumni.tugraz.at }

\maketitle

\begin{abstract}
We present an approach to enhance performance and flexibility of Bayesian
inference of model parameters based on observation of measured data.
Going beyond usual surrogate-enhanced Monte-Carlo or optimization
methods that focus on a scalar loss, we put emphasis on function-valued
input and output of formally infinite dimension. For this purpose,
the surrogate models are built on a combination of linear dimensionality
reduction and Gaussian process regression for the map between reduced
feature spaces. Since the decoded surrogate provides the full model
output rather than only the loss, it is re-usable for multiple calibration
measurements as well as different loss metrics and consequently allows
for flexible marginalization over such quantities. We evaluate the
method's performance based on a case study of a riverine diatom
model {[}1{]}. As input data, this model uses six tunable scalar parameters as well
as continuous time-series forcing data of weather and river discharge over a specific
year. The output consists of continuous time-series
data that are calibrated against corresponding measurements from the Geesthacht Weir
station at the Elbe river.
Results are compared to an existing model calibration using direct
simulation runs without a surrogate {[}2{]}.


\medskip\noindent References:

{[}1{]} M. Scharfe et al. Ecol. Model. 220, 2173-2186 (2009).

{[}2{]} U. Callies et al. Submitted to Environ. Model. Asses. (2021).

\medskip\noindent Key Words: Parameter inference, Monte Carlo, surrogate model,
Gaussian process regression, dimensionality reduction
\end{abstract}
\thispagestyle{empty}
\end{document}
